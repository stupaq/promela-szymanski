%% Generic short report include by Mateusz Machalica
%
\documentclass[a4paper]{article}

\usepackage[T1]{fontenc}
% this needs to be loaded before babel-polish
\usepackage{amssymb}
\usepackage{amsmath}
\usepackage{amsthm}
\usepackage[english]{babel}
\usepackage[utf8]{inputenc}
\usepackage[margin=2cm]{geometry}
\usepackage{graphicx}
\usepackage{ifthen}
\usepackage{hyperref}
\usepackage{algorithm}
\usepackage{algpseudocode}
\usepackage{wrapfig}
\usepackage{tikz}
\usepackage{float}

\renewcommand*{\arraystretch}{1.5}

\newcommand{\makeheader}[3]{
  \begin{center}
    \begin{tabular*}{\textwidth}{@{\extracolsep{\fill}}lr}
      #2 & \ifthenelse{\equal{#3}{}}{\today ~r.}{#3} \\[4pt]
      \multicolumn{2}{c}{{\Large #1}} \\[4pt]
      \hline
    \end{tabular*}
    \vspace{4pt}
  \end{center}
}
\newcommand{\floor}[1]{ \lfloor #1 \rfloor }
\newcommand{\ceil}[1]{ \lceil #1 \rceil }
\newcommand{\floorfrac}[2]{ \floor{\frac{#1}{#2}} }
\newcommand{\ceilfrac}[2]{ \ceil{\frac{#1}{#2}} }
\newcommand{\pair}[2]{ \langle #1, #2 \rangle }
\newcommand{\stirset}[2]{ \{ {#1 \atop #2} \} }
\newcommand{\stirperm}[2]{ [ {#1 \atop #2} ] }
\newcommand{\set}[1]{ \{ #1 \} }
\newcommand{\tuple}[1]{ \langle #1 \rangle }
\newcommand{\e}[1]{ \mathbb{E}\left[ #1 \right] }
\newcommand{\odd}[1]{ \operatorname{odd}\left( #1 \right) }
\newcommand{\sgn}{ \operatorname{sgn} }

\newtheorem{lemma}{Lemma}
\newtheorem{theorem}[lemma]{Theorem}
\newtheorem{remark}[lemma]{Remark}
\newtheorem{problem}[lemma]{Problem}
\newenvironment{solution}{\begin{proof}[Solution]}{\end{proof}}

\algblock{ForParallel}{EndForParallel}
\algrenewtext{ForParallel}[1]{\textbf{for} #1 \textbf{do in parallel}}
\algrenewtext{EndForParallel}{\textbf{end for}}

% vim: noet:sw=2:ts=2

\usepackage{multirow}
\usepackage{pifont}
\newcommand{\cmark}{\ding{51}}
\newcommand{\xmark}{\ding{55}}

\newcommand{\eventually}{\, \operatorname{\lozenge} \,}
\newcommand{\always}{\, \operatorname{\square} \,}
\newcommand{\until}{\, \operatorname{\mathcal{U}} \,}

\begin{document}
%%%%%%%%%%%%%%%%%%%%%%%%%%%%%%%%%%%%%%%%%%%%%%%%%%%%%%%%%%%%%%%%%%%%%%%%%%%%%%%%%%%%%%%%%%%%%%%%%%%%

\makeheader{Analysis of Szymanski's algorithm in Spin}{Mateusz Machalica}

\section{Analysis summary}

% TODO
\begin{center}
\begin{tabular}{|c|c|c|c|c|c|c|c|c|c|}
\hline
\multirow{4}{*}{properties}     & \multicolumn{9}{|c|}{models} \\
                                & \multicolumn{7}{|c|}{failure susceptible} & \multicolumn{2}{|c|}{failure resistant} \\
                                & never blocking & \multicolumn{6}{|c|}{blocking in local section} & never blocking & blocking in local section \\
                                & 321 & 321 & 312 & 231 & 213 & 132 & 123 &  &  \\
\hline
\ref{mutual_exclusion}          &  &  &  &  &  &  &  &  & \\
\hline
\ref{inevitable_anteroom}       &  &  &  &  &  &  &  &  & \\
\hline
 \ref{anteroom_exit}            &  &  &  &  &  &  &  &  & \\
\hline
 \ref{liveness}                 &  &  &  &  &  &  &  &  & \\
\hline
 \ref{linear_wait}              &  &  &  &  &  &  &  &  & \\
\hline
\end{tabular}
\end{center}

I strongly believe that most of the labels in above summary are self-explanatory.
The only model's feature which might be a bit cryptic is the permutation associated with it.
If we associate the following labels with assignment statements present in protocol's epilogue:
\begin{enumerate}
    \item \verb+chce[i] = false;+
    \item \verb+we[i] = false;+
    \item \verb+wy[i] = false;+
\end{enumerate}
, then the permutation denotes the order of execution of above statements in the epilogue.
For detailed description of experiments methodology, please see appropriate section.

\section{Introduction}

During the course of this brief analysis of Szymanski's mutual exclusion algorithm we will repeatedly refer to the following properties:
\begin{enumerate}
    \item\label{mutual_exclusion}
        $\always \forall_i \forall_j (i = j \lor \neg \verb+P[i]@critical_section+ \lor \neg \verb+P[j]@critical_section+)$
    \item\label{inevitable_anteroom}
        $\neg (\exists_i \eventually (\verb+P[i]@request_entry+ \until (\neg (\verb+we[i] && !chce[i]+) \until \verb+P[i]@critical_section+)))$
    \item\label{anteroom_exit}
        $\always \forall_i (\exists_j (\verb+we[i] && !chce[i]+ \implies i \neq j \land \verb+wy[j]+))$
    \item\label{liveness}
        $\always \forall_i (\verb+P[i]@request_entry+ \implies \eventually \verb+P[i]@critical_section+)$
    \item\label{linear_wait}
        linear wait % TODO
\end{enumerate}
, where we quantify over the set of process identifiers and constructs of the form $\verb+P[i]@label+$ mean that process $i$ is about to execute instruction labelled with \verb+label+.
Names of the labels should be understood intuitively and they match names of labels present in provided models.
Note that pairs of operators $\forall$ and $\always$ as well as $\exists$ and $\eventually$ are commutative, therefore all properties have the expected invariant form $\always (\ldots)$.

% TODO

\section{Experiments methodology}

% TODO

\section{Results discussion}

% TODO

\section{Final remarks}

% TODO

%%%%%%%%%%%%%%%%%%%%%%%%%%%%%%%%%%%%%%%%%%%%%%%%%%%%%%%%%%%%%%%%%%%%%%%%%%%%%%%%%%%%%%%%%%%%%%%%%%%%
\end{document}
% vim: noet:sw=2:ts=2:tw=160
